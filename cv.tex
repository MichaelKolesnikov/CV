\documentclass[14pt]{extarticle}
\usepackage{mystyle}

\begin{document}
\noindent
\begin{minipage}{0.35\textwidth} % Левая часть
      \section*{Колесников \\ Михаил \\ Леонидович}
      20 лет (19.09.2004)
      г. Москва
      \vspace{2cm}
      \subsection*{Контакты}
      \textbf{Email:} \href{mailto:Michael5Kolesnikov@yandex.ru}{Michael5Kolesnikov@yandex.ru} \\
      \textbf{Телефон:} \href{tel:+79850777240}{+7 (985) 077-72-40} \\
      \textbf{github:} \href{https://github.com/MichaelKolesnikov}{MichaelKolesnikov} \\

      \vspace{2cm}

      \subsection*{Навыки}
      \begin{itemize}
            \item Python
            \item C/C++/C\#
            \item html, css, js
            \item git
            \item linux
            \item php
            \item SQL, PostgreSQL, MySQL
      \end{itemize}

\end{minipage}
\hfill
\vline % Вертикальная черта
\hfill
\begin{minipage}{0.6\textwidth} % Правая часть
      \subsection*{Образование}
      \textbf{НИЯУ МИФИ} (09.2022 - 06.2026) \\
      Институт интеллектуальных кибернетических систем \\
      Направление подготовки:\\ «Программная инженерия»

      \subsection*{Дополнительное образование}
      \begin{itemize}
            \item 2024 г.
                  \subitem Тренировки по алгоритмам 5.0, Yandex
            \item 2023 г.
                  \subitem HTML CSS JS, Stepik
                  \subitem Алгоритмы: теория и практика. Методы, Stepik
      \end{itemize}
      \subsection*{Личные проекты}
      \begin{itemize}
            \item \textbf{Симуляция встречи работодателя и соискателей.} \\
                  Создание программы для симуляции встречи работодателя и соискателей, с целью оптимизировать процесс найма. \\
                  \href{https://github.com/MichaelKolesnikov/SimulationMeetingBetweenEmployerAndApplicants}{SimulationMeetingBetweenEmployerAndApplicants}

            \item \textbf{Графы.} \\
                  Исследование работы с графами, реализация различных алгоритмов на графах. \\
                  \href{https://github.com/MichaelKolesnikov/Graphs}{Graphs}

            \item \textbf{Структуры данных.} \\
                  Реализация различных структур данных для оптимизации работы программ. \\
                  \href{https://github.com/MichaelKolesnikov/Structures}{Structures}

            \item \textbf{PyStructures.} \\
                  Реализация структур данных на языке программирования Python. \\
                  \href{https://github.com/MichaelKolesnikov/PyStructures}{PyStructures}
      \end{itemize}
      \subsection*{Достижения}
      \begin{itemize}
            \item \textbf{Призер VI Научно-практической конференции "Шаг в науку".} \\
                  Весенняя научная сессия СНО НИЯУ МИФИ (2023 г.)

            \item \textbf{Призер Олимпиады "Физтех" по математике.} (2022 г.)

            \item \textbf{Призер Регионального этапа ВСОШ по информатике.} (2022 г.)

            \item \textbf{Финалист Олимпиады "True Tech Champ" от МТС.} (2023 г.)

            \item \textbf{Финалист Мегаолимпиады ИТМО.} (2023 г.)

            \item \textbf{И др.} Список всех достижений по ссылке: \\
                  \href{https://disk.yandex.ru/d/E6kgDRimXN08uQ}{https://disk.yandex.ru/d/E6kgDRimXN08uQ}
      \end{itemize}
\end{minipage}

\end{document}